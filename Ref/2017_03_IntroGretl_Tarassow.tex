%\documentclass[dvips]{beamer}
%\documentclass[10pt,xcolor=dvipsnames,mathserif]{beamer} % use option handout for handout
\documentclass{beamer}[11pt]
%\documentclass[handout]{beamer} %Pausen werden nicht berücksichtigt
%\documentclass[dvips,notes=onlyslideswithnotes]{beamer}
%\documentclass[pdftex]{beamer} %kompiliert jpeg, ABER LaTex to PDF

\usepackage[utf8]{inputenc}

\usepackage[scaled=1]{helvet}    %% --- Helvetica (Arial)
%\usepackage{courier} 
%\usepackage[pdftex]{graphicx}
%\usepackage{textcomp}
\usepackage{epstopdf}
\usepackage{natbib,amssymb,rotating,subfigure}
\usepackage{amsmath}
\usepackage{amsthm}
\usepackage{amsfonts}
\usepackage{amssymb}
%\usepackage{breqn}			%Use the dmath-environment to break math eq. automatically!
\usepackage{makeidx}
\usepackage{txfonts}
\usepackage{colortbl}
\usepackage{moreverb}
\usepackage{eurosym}
\usepackage{verbatim}
%\hypersetup{colorlinks,linkcolor=blue,urlcolor=links}
\usepackage[ngerman]{babel}
\usepackage{color}
\usepackage{dcolumn,booktabs}
\usepackage{graphicx}
\usepackage{caption}
%\usepackage{llncs}
\usepackage{supertabular}
%\usepackage{threeparttable}
\usepackage[flushleft]{threeparttable}


\newcommand{\subsize}[1]{\tiny{#1}}

% Tables
\usepackage{booktabs}
\usepackage{multirow}


%Templates
% 1
%\mode<presentation>
%{
%  \usetheme{default} % Singapore % Montpellier
%  \setbeamercovered{transparent}
%	\setbeamertemplate{footline}[page number]
%	\setbeamertemplate{navigation symbols}{}
%}
\mode<presentation>
{
	\usetheme{CambridgeUS}%{default} % Singapore % Montpellier
	\setbeamercovered{transparent}
	%\setbeamertemplate{footline}[page number]
	%\setbeamertemplate{navigation symbols}{}
	\useinnertheme{rectangles}
}

%\pgfdeclareimage[height=1cm]{logo}{ecbLogo}
%\logo{\pgfuseimage{logo}}
%\usecolortheme{beaverJ}


% 2
\setbeamertemplate{sections/subsections in toc}[square]
\setbeamertemplate{navigation symbols}{} % suppresses all navigation symbols:

% %\usetheme{Madrid}
%\usetheme{Singapore}
% %\usecolortheme[named=NavyBlue]{structure}
%\useinnertheme{rectangles}


%%%%%%%%%%%
\def\newblock{\hskip .11em plus .33em minus .07em}
\newcolumntype{d}[1]{D{.}{.}{#1}}



\definecolor{myRed}{rgb}{0.8,0,0}
\definecolor{jirkasblue}{rgb}{0.2,0.2,0.7}
\definecolor{jirkasred}{rgb}{0.9,0,0}
%\newcommand{\jemph}[1]{{\color{jirkasred}#1}}
\newcommand{\remph}[1]{{\color{myRed}#1}}
\newcommand{\rbsymbol}[1]{\jemph{\boldsymbol{#1}}}   % "rb" stands for redbold
\definecolor{myGray}{rgb}{0.2,0.2,0.2}
\definecolor{jirkasgray}{rgb}{0.8,0.8,0.8}
\newcommand{\jemph}[1]{{\color{jirkasgray}#1}}
%\newcommand{\remph}[1]{{\color{myGray}#1}}
\newcommand{\gbsymbol}[1]{\jemph{\boldsymbol{#1}}}   % "gb" stands for redbold


\title{\textsc{Using Gretl and Hansl for Econometrics}}
\subtitle{}

\author[Tarassow] % (optional, for multiple authors)
{Artur Tarassow\inst{1}\\
	\textsc{\small }
	}
 
\institute[] % (optional)
{
  \inst{1}%
  Faculty of Economics and Social Sciences\\
  University of Hamburg
}
 
\date[] % (optional)
{Melbourne, 23.03.2017}


%***********************************************************************************************
\begin{document}

\begin{frame}
  \titlepage
\end{frame}


\begin{frame}
  \frametitle{Outline}
  \tableofcontents%[pausesections]
\end{frame}


\section{Aims and general character}


\begin{frame}{Aims and general character}
	
	Main uses of econometric quantification:\footnote{The content of this presentation relies heavily on the joint work of Allin Cottrell (Wake Forest University) and Riccardo "Jack" Lucchetti (Università Politecnica delle Marche).}
	
	\begin{itemize}
		\item testing of theories
		\item forecasting
		\item policy analysis
	\end{itemize}
	
	\medskip
	
	\emph{Application} of economic and statistical theory to analysis of
	socio-economic data.
	
	\medskip
	
	But also \emph{development} of relevant statistical theory.
\end{frame}

\begin{frame}
	\frametitle{Dual status of econometrics}
	
	Econometrics is not a ``sub-field'' of economics in the same sense as, e.g., labour economics or health economics. 
	
	\medskip
	
	Rather it offers a set of \emph{tools} used in almost all sub-fields
	(other than pure theory) -- plus a discipline that is as much
	mathematical statistics as economics.
	
	\medskip
	
	However, an econometrician is not a statistician either: an
	econometrician is an economist with an above-average command of
	statistics.
\end{frame}


\section{Econometric software}

\begin{frame}
	\frametitle{Econometric coding}
	\begin{itemize}
		\item Pioneers in 1960s mostly used Fortran (some big names still
		do)
		\item In early PC era, command-line programs offering canned
		routines
		\item Gauss (1984, MS-DOS), Matlab (also 1984), Ox (around 1997)
		\item 1990s to date: Evolution of command-line programs: added GUIs
		and also elements of matrix-oriented languages (Stata, 1985;
		Eviews, 1994)
		\item recent tendency: packages built on top of matrix-oriented
		languages (Dynare)
	\end{itemize}
	
%	See, \emph{e.g.}, Charles Renfro, ``Econometric Software'', in
%	Belsley and Kontoghiorghes, \emph{Handbook of Computational
%		Econometrics} (2009).
\end{frame}

\begin{frame}
	\frametitle{Main numerical techniques in econometrics}
	\begin{itemize}
		\item Widespread use of matrices (mostly real, only rarely complex)
		\item Classical optimization techniques for smooth functions
		(fancier stuff like genetic algorithms also used, but sparingly)
		\item RNG (increasingly popular, especially for Bayesian techniques)
		\item A few concepts borrowed from engineering literature: spectra,
		filtering, signal extraction.
	\end{itemize}
	
	Traditionally, the dimensionality of problems is relatively small: a few Kb of RAM often suffice to hold data. This is rapidly changing nowadays with ``big data'' problems.
	
\end{frame}

\begin{frame}
	\frametitle{The ``market''/uses for econometrics-related
		software}
	
	\begin{enumerate}
		\item Undergraduate teaching (a big industry)
		\item Professional applied work; this could be split into
		\begin{enumerate}
			\item academic use (universities mostly)
			\item corporate use (traditionally large financial institutions but nowadays also large retailers such as Amazon/Google)
			\item policy (central banks, other government/supranational institutions)
		\end{enumerate}
		\item Development of new estimators/tools
	\end{enumerate}
	
	Currently dominated by proprietary software -- packages such as
	Stata and Eviews (for 1 and 2), plus Matlab (for 2.3 and 3).
	\medskip
	
	But	also incursion of R and gretl, and some use of lower-level languages for use 3.
	\medskip
	
	Very few people use compiled languages (C, Fortran etc.), even for CPU-intensive tasks.
\end{frame}


\section{Some background information on Gretl}

\begin{frame}{Some background information on Gretl \textbf{I}}
	\begin{itemize}
		\item Acronym for: 
		\textbf{G}nu \textbf{R}egression, \textbf{E}conometrics and \textbf{T}ime-series \textbf{L}ibrary
		\item URL: \url{http://gretl.sourceforge.net/}
		\item Comprises a large shared library, a common command line program and a GUI client.
		\item Makes use of reliable free software packages, e.g. (multi-threaded) LAPACK/BLAS, fftw, GTK, gnuplot, etc.
		\item First version released in January 2000 and has been under active development since then $ \to $ \textit{open-source} and \textit{free}.
		\item Written in C and available for Windows, OS X and Linux.
		\item User interface is available in 16 languages.
		\item Gretl is documented in a \textit{User Guide} of 350+ pages and a \textit{Command Reference} of 160+ pages; a tutorial introduction to hansl is also available.		
	\end{itemize}
\end{frame}

\begin{frame}{Some background information on Gretl \textbf{II}}
	\begin{itemize}
		\item Gretl comprises a full featured graphical user interface (GUI).
		\item Functions can be driven either by hansl scripting or by the GUI.
	\end{itemize}
	\vspace{0.5cm}
	\textbf{Unique Selling Point}
	
	\begin{itemize}
		\item Gretl offers a \textbf{high-level matrix oriented language} similar to Matlab and Gauss ...
		\item \textbf{AND} a high-level language that is attuned to \textbf{econometrics}.
	\end{itemize}
	
\end{frame}

\section{Datasets and matrices}

\begin{frame}{Datasets}
	\begin{itemize}
		\item A dataset is basically the union of matrix $ \textbf{y} $ (T by k) and $ \textbf{X} $ (T by m) with some additional metadata.
		\item Econometrics deals with the three main forms of data: (i) cross-section, (ii) time-series and (iii) panel.
		\item Think of a dataset as a big matrix. To make sense of a regression one needs to know:
		\begin{itemize}
			\item What the columns and rows refer to?
			\item Are the rows representing time period: (i) whas is the sample beginning and sample end, (ii) at what frequency were data recorded?
		\end{itemize}
		\item In econometric software the dataset is typically \textit{not} a matrix as such but a richer structure. \\
		$ \Rightarrow $ part or all of the dataset can be turned into a matrix on demand.
	\end{itemize}
\end{frame}


\begin{frame}{Datasets}
%	\begin{itemize}
%		\item \textbf{1st Duality} in hansl: the availability of the \textit{dataset}  as a specific data structure, alongside computer representations of the standard mathematical type.
%	\end{itemize}
	  Key abstractions in the econometric domain (besides those found in
	  general matrix-oriented software):
	  \begin{itemize}
	  	\item ``the dataset'' (plus series, lists of series)
	  	\item ``the bundle'' (as a means of avoiding the need for functions
	  	with an excessive number of parameters)
	  \end{itemize}
	  
	  %\pause
	  
	  \textbf{Duality}
	  \begin{itemize}
	  	\item Matrices, scalars, strings, arrays, bundles, functions (harder but flexible) \emph{versus}
	  	\item Datasets, series, commands (easy)
	  \end{itemize}
	  
	  \vspace{0.5cm}
	  
	  Softening of duality achieved via ``\underline{accessors}'' that can be used following commands. %(Non-trivial commands become somewhat function-like.)
	
\end{frame}

\section{Hansl}

\begin{frame}{Hansl -- “Hansl’s A Neat Scripting Language"}
	
	\begin{itemize}
		\item Path of gretl development similar to proprietary software (DOS
		command line program $\to$ cross-platform $\to$ add GUI $\to$ add
		Matlab-like matrix functionality $\to$ add advanced scripting $ \to $ parallelisation) %\pause
		\item ``Feel'' of scripting language owes something to bash shell (UNIX).
		\item C back-end (of course, with a little help from friends:
		netlib, BLAS, lapack, FFTW and others) %\pause
		\item Transition to development of gretl via hansl (function
		packages, with optional GUI integration) %\pause
		\item Some ``legacy'' formulations and inconsistencies, but hansl is much cleaner and easier to learn than many others (Stata, Eviews,	even R); OK, I may be a little biased here \texttt{;-)}
	\end{itemize}
\end{frame}


\begin{frame}{Hansl}
	\begin{itemize}
		\item Hansl's repertoire includes over 130 commands to date:
		\begin{itemize}
			\item (hypothesis) Tests
			\item (descriptive) Statistics
			\item Dataset (manipulation, sorting, etc.)
			\item Estimation (OLS, MLE, GMM, single-eq. and systems, etc.)
			\item Graphs (scatter plots, boxplot, time-series, etc.)
			\item Programming (control flow and debugging)
			\item Transformations
			\item Printing
			\item Utilities
			\item Forecasting
		\end{itemize}
	\end{itemize}
\end{frame}

\subsection{Models and methods}

\begin{frame}{Models}
	Gretl \textit{natively} implements a wide number of models and methods
	\begin{itemize}
		\item Time series methods
		\begin{itemize}
			\item ARIMA, univariate GARCH-type, (S)VARs and VECMs, unit root and cointegration tests, Kalman filter, MIDAS, real-time datasets
		\end{itemize}
		\item Limited dependent variables
		\begin{itemize}
			\item logit, probit, tobit, sample selection, interval regression, models for count and duration data, etc.
		\end{itemize}
		\item Panel-data estimators, including instrumental variables, probit and GMM-based dynamic panel models
	\end{itemize}
	Many more methods are provided by \textit{user-contributed} packages
	\medskip
	
	\url{http://ricardo.ecn.wfu.edu/gretl/cgi-bin/gretldata.cgi?opt=SHOW_FUNCS}
\end{frame}

\subsection{User-contributed provided packages}

\begin{frame}{User-contributed provided packages}
	Currently about 104 user-contributed packages are provided. Some time-series/panel packages you might be interested in are e.g.:
	\footnotesize
	\begin{itemize}
		\item \texttt{BMA} -- Bayesian Model Averaging for the linear regression models with jointness measures
		%\item \texttt{BNdecomp.} -- Beveridge-Nelson Decomposition
		\item \texttt{BreitungCandelonTest} -- Breitung-Candelon test of frequency-wise Granger (non-) causality
		\item \texttt{coint2rec} -- Cointegration stability tests (Hansen\&Johansen)
		\item \texttt{DIF\_panelcoint} -- Panel (non)-cointegration tests with bootstrap p-values
		\item \texttt{gig} -- An assortment of univariate GARCH models
		
		\item \texttt{gregory\_hansen} -- Residual-based tests for cointegration in models with regime shifts
		\item \texttt{johansensmall} -- Small-sample Johansen coint. rank tests (bootstrap and Bartlett)
		\item \texttt{StrucBreak} -- Autodetection of structural breaks in linear models, Bai-Perron style
		\item \texttt{SVAR} -- Structural VARs
		\item \texttt{Threshold\_Panel} -- Panel threshold model (Hansen, JoE 1999)
	\end{itemize}
\end{frame}


\section{Hansl by example}

\begin{frame}{Hansl by example: estimating a VAR}
	
	A \emph{VAR} is a commonly-used tool in macroeconomics: in its most
	basic incarnation:
	\begin{equation}
	\label{eq:VAR}
	y_t = \mu + \sum_{i=1}^p \Phi_i y_{t-i} + \varepsilon_t
	\end{equation}
	
	\begin{itemize}
		\item $y_t$ is an $n$-dimensional vector of observable quantities at
		time $t$; typically, macroeconomic variables: unemployment rate,
		inflation, interest rates, etc.
		\item The $\mu$ vector and the $\Phi_i$ matrices are unknown and
		must be estimated from the data.
		\item $\varepsilon_t$ is an $n$-dimensional vector white noise
		process.
	\end{itemize}
	
	Suppose that \eqref{eq:VAR} is a valid representation for the
	observed data: the econometrician's job is to estimate the unknown
	quantities so that the model can be used for policy analysis and
	forecasting.
\end{frame}

\begin{frame}
	
	\url{var-script.pdf}
	
	\url{var-script.inp}
	
\end{frame}



\section{Other aspects}

\begin{frame}{Supported data formats}
	Supported formats include: 
	
	\begin{itemize}
		\item Own XML data files
		\item Comma Separated Values
		\item Excel
		\item Gnumeric and Open Document worksheets
		\item Stata .dta files
		\item SPSS .sav files
		\item Eviews workfiles
		\item JMulTi data files
		\item Own format binary databases (allowing mixed data frequencies and series lengths)
		\item RATS 4 databases and PC-Give databases
		\item Includes a sample US macro database. See also the gretl data page. 
	\end{itemize}
\end{frame}


\begin{frame}{Communication with other software}
	Gretl can interact with other software packages.
	\medskip
	
	Easily send and receive datasets and matrices.
	\medskip
	
	Call other software via Gretl's \texttt{foreign-language} block.
	\medskip
	
	List of software supported:
	
	\begin{itemize}
		\item R
		\item Ox
		\item Octave
		\item Stata
		\item Python
		\item Julia
	\end{itemize}
\end{frame}


\begin{frame}[fragile]%{Gretl and R Example}
	\textbf{Gretl and R Example}
	\footnotesize
\begin{verbatim}

function list RStructTS(series myseries)
    smpl ok(myseries) --restrict
    sx = argname(myseries)
    foreign language=R --send-data --quiet
        @sx <- gretldata[, "myseries"]
        strmod <- StructTS(@sx)
        compon <- as.ts(tsSmooth(strmod))
        gretl.export(compon)
    end foreign
    append @dotdir/compon.csv
    rename level @sx_level
    rename slope @sx_slope
    rename sea @sx_seas
    list ret = @sx_level @sx_slope @sx_seas
    return ret
end function
# ------------ main -------------------------
open bjg.gdt
list X = RStructTS(lg)
\end{verbatim}
\end{frame}


\begin{frame}{Parallelization in Hansl}
	
	Via \textbf{OpenMP}
	
	\medskip
	
	Via \textbf{MPI}: Gretl has
	
	%\medskip
	
	{\footnotesize
		\begin{tabular}{l}
			\texttt{scalar mpisend(object x, int dest)} \\ 
			\qquad  -- send object \texttt{x} to node \texttt{dest}\\
			\texttt{object mpirecv(int src)} \\ 
			\qquad  -- receive an object from node \texttt{src} \\
			\texttt{scalar mpibcast(object *x [,int root])} \\ 
			\qquad  -- broadcast object \texttt{x} \\
			\texttt{scalar mpireduce(object *x, string op [,int root])} \\ 
			\qquad  -- reduce object \texttt{x} via \texttt{op} \\
			\texttt{scalar mpiallred(object *x, string op)} \\
			\qquad  -- reduce object \texttt{x} via \texttt{op}, all nodes \\
			\texttt{scalar mpiscatter(matrix *m, string op [,int root])} \\ 
			\qquad  -- scatter matrix \texttt{m} using \texttt{op} \\
		\end{tabular}
	}
	\medskip
	
	For details and examples see here: \url{https://sourceforge.net/projects/gretl/files/manual/gretl-mpi-a4.pdf/download}
\end{frame}


\section{Get started with Gretl and Hansl}

\begin{frame}{Get started with Gretl and Hansl}
	Take a look at the following resources to start
	\footnotesize
	\begin{itemize}
		\item \textit{Gretl Command Reference} (accessible via the Help menu in the gretl GUI): contains a complete listing of the commands and built-in functions available in hansl, with a full account of their syntax and options.
		\item \textit{Gretl User’s Guide}: Chapters 10 to 16 on data types, loops, the definition and use of functions.\\
		\url{http://sourceforge.net/projects/gretl/files/manual/}
		\item \textit{Sample scripts}: The gretl package comes with a large number of sample or practice scripts (under menu item /File/Script files/Practice file).
		\item \textit{Function packages}: Ambitious examples of hansl coding (via the gretl menu item /Tools/Function packages/On server.)
		\item \textit{gretl-users mailing list}: Most well-considered questions get answered quite quickly and
		fully\\
		\url{http://lists.wfu.edu/mailman/listinfo/gretl-users}
		\item Textbook \textit{Using gretl for Principles of Econometrics} (4th edit.) \url{http://www.learneconometrics.com/gretl/}
	\end{itemize}
\end{frame}





%%%%%%%%%%%%%%%%%%%%%%%%%%%%%%%%%%%%
%%%%%%%%%%%%%%%%%%%%%%%%%%%%%%%%%%%%%%%%%%%%%%%%%%%%%%%%%%%%%%%%%%%%%%%%%
\section*{Referenzen}
\bibliographystyle{chicago}
\setcounter{secnumdepth}{0}

%\begin{frame}[allowframebreaks]
%	\frametitle{Literatur}
%	\footnotesize
%	\bibliography{../../../Dropbox/05_Literatur_DATENBANK/library}
%\end{frame}


%\section{Appendix}



\end{document}